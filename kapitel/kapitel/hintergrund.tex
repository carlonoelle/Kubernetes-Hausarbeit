\newpage
\section{Der Hintergrund}
In der Softwareentwicklung gibt es irgendwann den Punkt, bei dem man eine fertige Website oder App programmiert hat. Nun gilt es, dieses Produkt erreichbar zu machen, endweder nur ausgewählten Personen oder gleich der gesamten Öffentlichkeit.
Diesen Schritt nennt man Bereitstellung, oder auch zu Englisch Deployment. In dieser Arbeit fokussieren wir uns auf eine Website, die der Öffentlichkeit zugänglich gemacht werden soll. 

Diese fertige Website soll über das öffentliche Internet, auch WWW (World-wide-web), von jedem internetfähigen Gerät abrufbar gemacht werden. Dazu benutzt man einen Computer,
der die Website bereit hält um sie an Endgeräte auszuliefern. Man nennt solch einen Computer Server, und Endgeräte sind die Clients. Server für Websiten nennt man Webserver.

Ein solcher Server hat jedoch harte Grenzen was die Kapazität betrifft. Man spricht von Last auf den Server, welche durch zu viele Clients die gleichzeitig auf der Website sind erzeugt wird.
Wann diese Grenzen erreicht sind, hängt davon ab, auf wie viel Hardwareressourcen der Server zurückgreifen kann, also Arbeitsspeicher (RAM) und Prozessor- (CPU) Leistung. Der Server muss die Anfragen
verarbeiten können die er erhält, und wenn die Ressourcen schon durch andere Arbeitsschritte ausgelastet sind kann der Server neue Anfragen nicht mehr entgegennehmen.

Eine nicht stark frequentierte Website braucht keinen starken Server, jedoch besteht immer die Möglichkeit, das durch verschiedene Faktoren der sogenannte Traffic, die Menge Daten die über das Netzwerk geschickt werden, rapide Ansteigt und 
somit der Server überlastet wird. Durch nicht beantwortete Anfragen kann bei einem Onlineshop beispielsweise Geld verloren gehen, was natürlich nicht passieren soll.

Ein anderes Problem ist eigendlich ein ganz simples. Wenn die Hardware auf dem der Server läuft einen Defekt hat, ist die Website gar nicht mehr erreichbar.
Um dieses Problem zu umgehen dupliziert man einfach den Server auf eine andere Hardwarebasis. Somit hat man zwei Server, die die gleiche Website hosten (bereit halten).

Nur funktioniert das Internet so, dass eine Domain (Url, z.B.: beispiel.de) auf eine Adresse im Internet zeigt. Solch eine Adresse nennt man IP, und sie ist einzigartig im gesamten Netz.
Da wir aber jetzt auf eine Adresse zeigen, können somit nicht zwei Server antworten, sondern man landet immer bei dem der die Adresse trägt.
Dafür gibt es dann Reverse-Proxy`s, oder auch Load-Balancer. Im Grunde machen die nichts anderes, als eine Adresse zu haben, und die ankommende Anfrage an definierte Ziele weiter zu leiten.

Der Name Load-Balancer nimmt den positiven Nebeneffekt schon vorweg, den solch ein Proxy kann eine Anfrage nicht bloß zufällig an eine von n vielen Adressen weiterleiten, sondern auch nach verschiedenen Faktoren agieren.
Beispielsweise kann der Load-Balancer sich merken, wie viele Verbindungen gerade auf einem Server laufen, um neue besser an einen Server weiter zu leiten der weniger zu tun hat. Somit sind die zwei Hauptprobleme beseitigt.

Das soeben erklärte Prinzip nennt man Horizontale Skalierung, da ein Server neben den anderen gestellt wird (Horizontal), um somit mehr Last abfangen zu können (Skalierung).
Vertikale Skalierung ist sehr ähnlich, man verteilt die Last auf mehrere Server, jedoch nicht auf verschiedene Hardware. Man installiert z.B. zwei Webserver auf einem Computer. 
Dies kann einen Vorteil bringen was die Lastverträglichkeit angeht, da manchmal eine Software nicht die kompletten Hardwareressourcen alleine benutzen darf, jedoch bring es keine Ausfallsicherheit.

Durch Horizontale Skalierung erhält man also Ausfallsicherheit, da die Website redundant verfügbar ist, und die Last wird gleichmäßig verteilt auf alle Maschinen, wobei jederzeit weitere eingegliedert werden können.
Ausfallsichereit wird auch Hochverfügbarkeit genannt, da selbst im Falle eines Hardwaredefekts die Website immernoch erreichbar wäre. Wahre Hochverfügbarkeit würde erreicht werden, wenn die Replikate des Servers noch
an verschiedenen physikalischen Standorten stehen würden, wodurch man Faktoren höherer Gewalt wie Stromausfällen davon kommt. Infrastruktur, so wird die Hardware/Software Lösung hinter der Website/App genannt, ist ein großes
Thema und es gibt viele verschiedene Ansätze die gleichen Probleme zu lösen.

Die Planung, Ausführung sowie Instandhaltung und Pflege von Infrastruktur fällt in den Aufgabenbereich eines Systemintegraten. Zumindest ist dies die klare Trennung zwischen 
dem Softwareentwickler, also dem Programmierer, und der anderen Hälfte, dem Bereitstellen. Im weiteren Verlauf dieser Hausarbeit werden Lösungswege vorgestellt in denen die Aufgabenverteilung etwas verschmilzt.

Dieses Hintergrundwissen sorgt nun für Verständnis der folgenden Themen. 