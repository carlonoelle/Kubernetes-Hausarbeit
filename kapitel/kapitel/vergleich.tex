\newpage
\section{Chancen und Risiken}
Mit jedem der genannten Konzepte wird das Ziel verfolgt eine App zu starten sowie diese dann erreichbar zu machen, der Öffentlichkeit oder ausgewählten Personen.
Schon genannte Aspekte die zu beachten sind währen die Ausfallsicherheit sowie Einfachheit des Arbeitens.

Die erklärten Möglichkeiten der Skalierung währen mit allen Konzepten anwendbar, jedoch wird klar das mittels der Kontainerisierung dies am schnellsten machbar währe.
Während bei Bare-metal oder Virtualisierung ein weiterer Server/VM erstellt und eingerichtet werden muss, muss bei Docker nur ein Befehl ausgeführt werden um innerhalb Sekunden eine zweite 
Instanz der App hochzufahren.
Darüber hinaus ist ein wichtiger Faktor das bei den Wegen ohne Docker Fehler passieren können. Fehler, die auch vielleicht erst irgendwann während der Benutzung auffallen.
Zum Beispiel wenn die falsche Version einer Abhängigkeit installiert wird, oder Dateien an die falsche Stelle kopiert werden.

Diesen Faktor der Fehleranfälligkeit ist bei der Containerisierung komplett irrelevant, denn ein Docker Image ist unveränderbar, und wenn es gestartet wird 
entstehen immer exakte Kopien.

Zeit ist in jedem Unternehmen eine kostbare Währung. Wie bereits erwähnt zieht man mittles Kontainerisierung in Sekunden eine neue Instanz jedlicher Server hoch.
Dies ist mittlerweile ähnlich bei Virtualisierung auch möglich, dort wird mit Kopien von VM's gearbeitet. Bedeutet, das eine VM mit Betriebssystem präperiert wird, und
sobald dann eine weitere Instanz mit dem OS benötigt wird dupliziert man die vorgefertigte Maschine. Dadurch spart man den Schritt der Neuerstellung und Einrichtung, jedoch
ist dies in keinster im Sekundenrahmen, und in den meisten Fällen muss die VM die kopiert werden soll ausgeschaltet sein damit sie kopiert werden kann. 
Dies würde bei einer Website bedeuten das man sie offline nehmen muss, um die Kopierfähigkeit auszunutzen.
Bare-Metal hat kaum Möglichkeiten gut ein System zu duplizieren der nicht mit viel händischer Arbeit verbunden wäre.

Soviel dazu, nur muss ja wie schon bereits erwähnt muss um Skalieren zu können ein Proxy vor allen verfügbaren Instanzen stehen, damit eine Adresse auf alle zeigen kann.
Dieses Prinzip trifft auf alle Bereitstellungskonzepte gleich zu, jedoch wurde mittels der Software Kubernetes diese Arbeit erheblich erleichtert. Auch die Skalierung
wird auf eine neue Ebene gehoben.

Nun sind die hauptsächlichen Chancen und Risiken der verschiedenen Platformen bekannt, jedoch hat die Containerisierung aufgrund ihrer 
unveränderlichen Images sowie Geschwindigkeit noch weitere Möglichkeiten die und die anderen Platformen nicht so leicht bieten können.

\subsection{Kubernetes}
